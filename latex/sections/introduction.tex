\begin{document}

Temperature measurements are made all over the world every day. The data is used for a lot of different purposes, e.g. weather broadcasts, but the one we all think about is probably confirming global warming and the greenhouse effect. 

SMHI (Swedish Metrological and Hydrological Institute) measures the temperature on various locations. This data can be found on their website \cite{smhidata}. The temperatures are measured between one and three times per day. Several locations has datasets that started about 60 years ago (e.g. Lund), but the Uppsala dataset contains average daily temperatures since 1722. In this report we analyze this data in ROOT and present the results. These results are: the temperature of a given day, and the beginning of the seasons, the warmest and coldest day of each year, the extrapolation of the mean temperature of each year . 

As mentioned above, ROOT is used for the data analysis, which is a scientific software written in C++. Our code is structured in several different parts, with a class that contains all our functions. The common functions are: read data, print data, detect leap years and one that returns the day number given the date. In addition, each of the four produced results has a separate .cpp-file. 

\end{document}
